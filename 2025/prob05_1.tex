\newlinechar=`\^^J
\message{^^J^^J}

% Since TeX does not support 64-bit numeric values, we need to use
% string comparison. Hereare helpers to zero-pad strings to 20 digits.
% https://tex.stackexchange.com/questions/501660/how-to-count-number-of-digits-in-an-integer-or-letters-in-a-word-and-then-retr
\def\fullyexpand#1{\romannumeral - `0#1}
%\def\fullyexpand#1{\edef\fetemp{#1}\fetemp}

\def\zCountDigits#1{%
  \number\numexpr0\expandafter\zCountDigitsAux#1\zCountDigitsEnd\relax
}
\def\zCountDigitsAux#1{%
  \ifx\zCountDigitsEnd#1\else+1\expandafter\zCountDigitsAux\fi
}
\def\zCountDigitsEnd{\zCountDigitsEnd}

% this craziness allows \pad to work with a macro or literal
\def\dopad#1\relax{\ifnum\zCountDigits{#1}<20 \pad{0#1}\else#1\fi}
\def\pad#1{\expandafter\dopad#1\relax}

\newread\file
\openin\file=data05.txt

% test for ranges
\def\checkinrange#1#2{
  \count1=\pdfstrcmp{\subject}{#1}
  \ifnum\count1<0
  \else
    \count1=\pdfstrcmp{\subject}{#2}
    \ifnum\count1>0
    \else
      \advance\count0 by 1
    \fi
  \fi
}

\def\ranges{\count0=0}
\def\newrange#1#2{
  \message{new range #1 to #2^^J}
  \expandafter\def\expandafter\ranges\expandafter{\ranges\checkinrange{#1}{#2}}
}
\def\split#1-#2 {\newrange{\pad{#1 }}{\pad{#2 }}} % the space is important!
\def\mysplit#1{\expandafter\split#1\relax}

\def\ifinranges#1{
  \def\subject{\pad{#1}}
  \message{\subject}
  \ranges
  \ifnum\count0>0
}

\message{Parsing fresh ranges...^^J}

% Negative version of loop to read lines from a file.  Made it "long"
% to allow \par in the conditional so we don't need to hide that in a
% separate macro. (Is it really this hard?)
\long\def\nloop#1\repeat{\def\body{#1}\niterate}
\def\niterate{\body \let\next=\relax \else\let\next=\niterate\fi \next}
\let\repeat=\fi % this makes \loop...\if...\repeat skippable

% read to blank line
\def\ifisnum#1{\ifcat1#1}
\nloop
  \read\file to\fileline
  \ifisnum\fileline\mysplit\fileline\fi
  \ifcat\par\fileline % if the line is blank
\repeat

\message{Parsing available ingredients...^^J}

\count2=0
\def\checknum#1{
  \ifinranges{#1}\message{fresh^^J}\advance\count2 by 1\else\message{^^J}\fi
}

% read to end
\nloop
  \read\file to\num
  \ifisnum\num
    \checknum{\num}
  \fi
  \ifeof\file
\repeat

\closein\file

\message{Sum: \the\count2^^J^^J}

\end
